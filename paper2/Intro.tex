% \section{Introduction}

Turbulent boundary layers (TBLs) play an essential role in a wide range of areas, including atmospheric flows, aircraft design or turbine blades. Assessing the effect of streamwise pressure gradients on TBLs is critical to completely understand these applications \cite{harun_monty2013, Maciel_2018}, and this is very challenging due to the complex effect of flow history on the local state of turbulence \cite{bobke2017, tanarro_2020, Kitsios2017}. Here we study adverse-pressure-gradient (APG) effects on TBLs on statistically two-dimensional flows subjected to a nearly-constant APG magnitude leading to near-equilibrium conditions \cite{Marusic_McKeon_2010, Nagib_2008}.
The turbulent character of the flow is analysed through the Reynolds decomposition \cite{Rey_decomp} of the fluid variables into a mean flow (averaged in time and the homogeneous spanwise direction) and a fluctuating component.
Through this statistical approach, the mean flow has been extensively studied and some universal laws have been established, such as the law of the wall \cite{vonKarman1931}, the defect law and log-law  \cite{millikan39}, together with various approaches to account for different flow configurations \cite{Luchini_2017}. 
One important tool for mathematical modeling of physical systems is scaling, which in the context of turbulent flows has been used for assessing the physics at higher Reynolds numbers~\cite{Hultmark_2012}, and also in combination with symmetry analyses~\cite{Oberlack_2022}.
The approach taken in this work is based on scaling different energetic regions of the boundary layer at high Reynolds numbers, and it can be useful for other multi-scale systems where there are different energy scales interacting with each other. In these cases, a marginal integration of the scales of various sizes can lead to understanding the relative relevance and sensitivity of the various scales.

In this study we assess the fluctuating components of APG TBLs, denoted by Reynolds stresses (RSs), where the streamwise normal component $\overline{u^2}$ is the most energetic. Note that $\overline{(\cdot)}$ denotes the average in time and the homogeneous spanwise direction.

In APG TBLs we deal mainly with two parameters: the Reynolds number and the APG magnitude, where the latter can also be a function of the streamwise position.
% \rev{It is interesting to find scaling factors for certain characteristics of the flow such that they can collapse with Reynolds number and/or APG magnitude, so as to further understand the physical behavior of the flow.}
Kitsios {\it et al.}~\cite{Kitsios2016, Kitsios2017} used an integral approach to self-similarity in order to obtain scaling factors for the Reynolds stresses, and compared the spectra of a ZPG and an APG. 
The integral scaling factor they used was the displacement thickness, which is defined as $\delta^*=\int_0^{\delta_{99}} (1 - U/U_{e}) \mathrm{d}y $, where $U$ is the mean streamwise velocity and $U_e$ is the velocity at the edge of the boundary layer ($y=\delta_{99}$). Note that $x$, $y$ and $z$ are the streamwise, wall-normal and spanwise coordinates, respectively. 
% \rev{The outer peak of all the RSs was shown to be at $y \approx 1.2\delta^*$ in Kitsios {\it et al.}~\cite{Kitsios2016} for their mild-APG simulation and at $y \approx \delta^*$ for their strong-APG case~\cite{Kitsios2017}. }
Following this RS scaling, Maciel {\it et al.}~\cite{Maciel_2018} and Sanmiguel Vila {\it et al.}~\cite{Sanmiguel_PRF} scaled the outer peaks of the RSs with $\delta^*$ and $\delta_{99}$ for several numerical and experimental databases with a large range of APG magnitudes. 
% \rev{ The former focused on non-equilibrium databases, such as the APG TBLs around wing profiles, and the latter showed results for experimental near-equilibrium TBLs developing on flat plates. }
In these studies, it was found that the wall-normal location of the outer peak of the RSs scales with $\delta^*$, making it a scaling independent of the Reynolds number and APG effects. 

Note that Maciel {\it et al.}~\cite{Maciel_2018} reflected on the $\delta^*$ scaling of the wall-normal location of the outer peak, $y_{\rm OP}$. Since $\delta^*/\delta_{99}$ decays slowly to zero when the Reynolds number tends to infinity, this would imply that the outer peak approaches the wall, therefore $\delta^*$ would only scale $y_{\rm OP}$, but not necessarily the size of the outer region.

Despite the different values for the wall-normal location of the outer peaks in the RSs for moderate and strong APGs documented in Ref.~\cite{Kitsios2017}, the spectral outer peak for the ZPG and the strong-APG cases were shown to be at the same wall-normal location: $y \approx \delta^*$. 
% \rev{
% Furthermore, the spanwise scale of the strong APG outer peak was reported to be at $\lambda_z \approx 2\delta^*$.
% }
In other studies where $\delta_{99}$ was used for scaling the outer region of the TBL \cite{Lee2017, bobke2017}, the outer-spectral-peak wavelength $\lambda_z$ appears to scale better in both ZPG and APG with $\delta_{99}$ than with $\delta^*$. 

% \rev{
% Most of the previous studies focused on short regions of interest and/or low Reynolds numbers, and the studies reaching higher Reynolds numbers were either non-equilibrium simulations or a result of a strong APG, resulting in a very pronounced development of the TBL. On the other hand, Pozuelo {\it et al.}~\cite{Pozuelo_JFM_22} conducted a well-resolved large-eddy simulation (LES) of a moderate-APG TBL which develops from ZPG conditions into a near-equilibrium APG with nearly-constant pressure-gradient magnitude. This unique database exhibits a long region of near-equilibrium flow at high Reynolds number, which enabled assessing the behavior of the near-wall and outer peaks of the streamwise Reynolds stress $\overline{u^2}$.
% }
Based on the work by Pozuelo {\it et al.}~\cite{Pozuelo_JFM_22}, in APG TBLs the wall-normal location and magnitude of the near-wall peak, denoted here by $y_{\rm IP}$ and $\overline{u^2}_{\rm IP}$ respectively, were found to increase with Reynolds number and APG magnitude using inner scaling. In this scaling, the viscous length $\ell_{\tau}=\nu/u_{\tau}$ and the friction velocity $u_{\tau}$ are employed, where $\nu$ is the fluid kinematic viscosity and $u_{\tau}=\sqrt{\tau_w/\rho}$ (where $\tau_w$ is the wall-shear stress and $\rho$ is the fluid density). The superscript `+'  will be used to denote inner scaling.
Regarding the outer peak, its wall-normal location $y_{\rm OP}$ was found to be approximately constant when scaled with either the displacement thickness $\delta^*$ or the boundary-layer thickness $\delta_{99}$, since for a moderate APG $\delta^*/\delta_{99}$ decreases slowly with the Reynolds number.
With the former length scale, $y_{\rm OP}/\delta^*$ appeared to be less influenced by the APG, obtaining an approximate value of 1.4, while using $\delta_{99}$, the locations varied with APG magnitude. 

% \rev{
% On the other hand, the ZPG case exhibits some energy in the outer region which did not develop into an outer peak of $\overline{u^2}$; however, the power-spectral density of this quantity exhibits an outer peak similar to that of the APG case.
% }
Analyzing the RS using the spanwise spectra with different scaling factors, we will assess similarities and differences with respect to the ZPG and the scaling properties of the RS. This has important implications in the context of the fundamental structure of turbulence.


