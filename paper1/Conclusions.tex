\section{Summary and conclusions} \label{sec:Conclusions}

A new well-resolved LES of a TBL developing on a flat plate subjected to an adverse pressure gradient is presented. The relevance of this simulation lies in the high Reynolds number we achieve starting from a laminar flow under similar conditions to those in experiments, and obtaining a long region where the TBL is in near-equilibrium. 
To the authors’ knowledge, this is the first TBL under approximately-constant APG magnitude  ($\beta\approx 1.4$) over a long near-equilibrium region up to $\Rey_{\theta}=8700$. The characteristics of this simulation have enabled a direct comparison with experimental data in a similar range of $\beta$ and $\Rey_{\tau}$, as well as the comparison with other near-equilibrium databases at lower Reynolds number as discussed below. 
The data obtained from the simulation consists of two-dimensional turbulence statistics for all the grid points in the $xy$ plane, together with time series and two-point correlations at 20 streamwise locations (containing all the spanwise grid points), with 10 of those profiles in the near-equilibrium ROI.

The turbulence statistics were compared with lower-Reynolds-number simulations of different APGs \citep{bobke2017}, with another high-Reynolds-number well-resolved LES of a ZPG \citep{E-AmorZPG} and also with high-Reynolds-number experiments \citep{MTL_expSANMIGUEL} with a similar $\beta(x)$ development. 
These comparisons highlight the need of more simulations with long near-equilibrium regions to be able to distinguish the effects of the APG and the effects of the Reynolds number. The near-equilibrium features have been analysed with the Rotta-Clauser pressure-gradient parameter $\beta$ and the parameter $\Lambda_{\rm inc}$ for different sets of velocity and length scales, {\it i.e.} the edge and the Zagarola--Smits scalings. Near-equilibrium conditions were obtained in the region from $\Rey_{\tau}=800$ to around 2000. 
The results were also compared to another constant-$\beta$ database \citep{Kitsios2016}, and we showed that any self-similarity analysis has to be performed along regions of near-equilibrium at high Reynolds numbers  to be able to study the collapse of the different regions in the TBL.
For the APGs under study here, the viscous scaling collapses the near-wall region for the streamwise mean velocity and although the magnitude of the near-wall peak of the streamwise RS increases in inner scaling, its location remains close to $y^+ \simeq 15$.
Furthermore, the outer scaling shows a good collapse of the outer region.

The large scales associated with the APG have an effect in the inner and outer regions, similarly to the large scales present in the flow at high Reynolds numbers in simpler geometries, such as channel flow \citep{Hoyas_PoF2006}. This was assessed for the various terms of the Reynolds-stress tensor using the spanwise one-dimensional power-spectral density and the two-dimensional power-spectral density in spanwise spatial scales and the temporal scales. 
This study shows that the displacement of small scales with $\lambda_z^+ \simeq 100$ from the inner to the outer region, documented in APGs at low Reynolds numbers by \cite{tanarro_2020, VINUESA2018}, is also observed in high-$\Rey$ APGs.
The present analysis at higher Reynolds numbers shows that there is an energization associated not only with the small scales but also with longer spatial and temporal scales in the streamwise and spanwise Reynolds stresses near the wall and in the overlap region.
The current database enables spatio-temporal analysis over a wide range of Reynolds numbers in near-equilibrium conditions due to the long near-equilibrium region at high $\Rey$. 
This type of high-quality APG TBL in near equilibrium is necessary to deepen our insight into the similar but different effects of Reynolds number and APG, and to further understand the role of flow history on the local features of wall-bounded turbulence 



