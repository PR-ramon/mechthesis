% \section*{Appendix \hypertarget{AppB}{B}}\label{sec:AppendixB}
\section*{Appendix B} \label{sec:AppendixB}

Turbulent flows are characterised by fluctuations of the flow variables, which may have different amplitudes and frequencies, and are linked with a wide range of eddy structures of different spatial and temporal scales.

As the TBL develops and the Reynolds number increases, the range of scales of the eddies grows, and a finer grid is needed to properly resolve all the scales. In this work, we implement a well-resolved LES which resolves the larger eddies in a suitable grid and a subgrid-scale (SGS) model is used to take into account the effects of the smaller scales. The SGS model used in this study is the approximate deconvolution model with a relaxation-term (ADM-RT), which is further documented in \cite{Schlatter_2004}. 
This model does not involve an eddy viscosity, and it is based on filters applied on the equidistant grids in Fourier space and on the non-equidistant wall-normal direction. As stated in \cite{Schlatter_2004}, the model does not disturb the flow development as long as it is still sufficiently well resolved, otherwise it adds the additional necessary dissipation.
The relaxation term can be seen as an SGS force of the form:
\begin{equation*}
    \frac{\partial \tau_{ij}}{\partial x_j} = \chi H_N * \overline{u}_i,
\end{equation*}
which is added to the right-hand side of the filtered Navier--Stokes equations.
The coefficient $\chi=0.2$ is proportional to the inverse of the time step of the integration. The high-order filter $H_N$ uses a cutoff frequency $\omega_c \in (0, \pi]$ (in this simulation $\omega_c=2\pi/3$), which only affects the smallest scales. The filter is applied through a convolution (denoted by the symbol $*$) to the velocities $\overline{u}_i$. The velocities are marked with an overline to indicate that they are implicitly filtered because of the lower resolution of the LES grid.


This SGS model has been compared with DNS simulations on multiple occasions: for transitional channel flow in \cite{Schlatter_2004}, for ZPG TBL in \cite{E-AmorZPG} and for turbulent wings in \cite{NEGI_2018}. The filter has been used in the APG cases by \cite{Bobke_2016, bobke2017}.
\cite{E-AmorZPG} reports for $\Rey_{\tau}=3600$ , $87.2\%$ of the dissipation of the DNS being resolved by the LES, while the addition of the SGS contributed to recover $99.8\%$.



 
