\section{Introduction} \label{sec:Introduction}
%----------------------------------------------------
The study of boundary layers (BL) is the study of the behaviour of a fluid close to the boundary with another substance (solid, liquid or gas). We can use the knowledge of these boundary layers to predict the weather (systems atmosphere-Earth or atmosphere-sea) or even to manipulate a fluid for engineering purposes (production of electricity, mixing processes, transportation,...); note that most of these cases exhibit turbulent motions. In wall-bounded flows, as in the previous engineering examples, the optimization and control of the boundary layers is crucial for a good performance of the application under study. Some of the typical objectives are to reduce the drag in aeronautical surfaces, control the transition to turbulence or to avoid recirculation bubbles in ducts where achieving the maximum mass flow rate is important.

All the turbulent boundary layers (TBLs) in practical applications are subjected to streamwise pressure-gradients (PG), often with complex streamwise PG histories. These complicated variations of the PG can be divided into regions of adverse (APG), favourable (FPG) or zero pressure gradient (ZPG). The effects of PGs in wall-bounded turbulence are very diverse: a FPG drives the flow in a turbulent channel, while (if strong enough) it can produce relaminarization in flat-plate TBLs \citep{narasimha_sreenivasan_1973, FPG_araya2015}. On the opposite side, an APG can promote turbulence in a laminar BL and increase the turbulent fluctuations of a turbulent boundary layer; it can even produce flow separation, which is an undesirable phenomenon that reduces the performance of an aerodynamic device and can be dangerous if it happens on the wing of an airplane.
In wall-bounded flows the BL is affected by the wall geometry, the characteristics of the wall surface, the pressure-gradient distribution and the flow state beyond the BL. The effects on the wall will be seen as fluid-dynamic forces (lift and drag), but we could also be interested in effects produced after the solid, {\it i.e.} in the  the wake and its aeroacoustics properties (noise), or the wake instabilities that produce an increase in time between take-offs in airports.

Understanding the energy-transfer mechanisms within the TBL, and how they are affected by the pressure gradient, may lead to advancements in flow control and to new aerodynamic designs with higher performance.
The complexity of the problem has led to the definition of canonical TBLs in simple geometries, such as flat plates subjected to a ZPG or PG TBLs in equilibrium or near-equilibrium. 

It is important to discuss the concept of equilibrium and the effects of flow history. As discussed by \cite{Gibis2019} and \cite{Marusic_PoF_2010}, the term `equilibrium' has been used in different contexts in the literature. \cite{Clauser_1954_exp} denoted ``equilibrium profiles'' as those profiles of a boundary layer which develop maintaining a non-dimensional ``constant history'', defined as a constant ratio of the pressure-gradient and wall-shear forces, which can be written as the Clauser pressure-gradient parameter $\beta=\delta^{*}/\tau_w (\textrm{d}p/\textrm{d}x)$, where $\delta^{*}$ is the displacement thickness, $\tau_w$ is the shear stress at the wall, and $(\textrm{d}p/\textrm{d}x)$ is the pressure gradient. 
Later, the term `equilibrium' was used by \cite{rotta1962turbulent} and  \cite{townsend_1956_eqBL} as a synonym of self-preserving flow and in \cite{townsend_1961} it is used to refer to regions of the flow (`equilibrium layers') where there is a balance between rates of energy production and dissipation. In this context we will follow the same criterion as  \cite{Marusic_PoF_2010}, avoiding the use of equilibrium boundary layers and referring to near-equilibrium TBLs when the mean velocity-defect $U_{e}-U$ in the outer layer exhibits self-similarity (note that $U$ is the mean streamwise velocity and $U_e$ the velocity at the boundary-layer edge).

After these studies, many PG TBL experiments and simulations have been performed, with the aim of obtaining constant-$\beta$ distributions in order to obtain PG TBLs with a well-defined flow history, in this case with constant PG magnitude. 
If the equation of momentum in the streamwise direction is integrated across the boundary layer under several assumptions, then the momentum integral equation will relate the evolution of the momentum thickness $\theta$ with $\delta^*$ and $\beta$.

From he momentum integral equation, it is possible to derive the argument that $\beta$ must be constant to obtain self-similarity based on integral assumptions.

However, note that the ZPG TBL exhibits at least two different scalings: the near-wall region which scales properly using viscous units and the outer region, which does it with outer units (such as $\delta^*$ or the boundary layer thickness).

Therefore, the interest in establishing well-defined and close-to-constant $\beta$ distributions is not related to integral self-similarity arguments but lies in the fact that it is an integral parameter that quantifies the ratio between the pressure forces and the friction forces the boundary layer is subjected to throughout its evolution.

If this ratio had strong variations, especially at low Reynolds numbers, then it would be difficult to understand whether the observed phenomena are due to a local effect of the pressure gradient or the result of many different ratios of friction/pressure forces.

In this sense, a constant-$\beta$ configuration, in principle, allows to separate pressure-gradient and Reynolds-number effects, by comparing TBLs that only differ in the magnitude of the ratio between friction and pressure forces, since that ratio is maintained during the evolution of each BL, therefore, not mixing the contribution of flow history.

An example of constant $\beta$ are ZPG TBLs, which have been widely studied numerically e.g. by \cite{schlatter_orlu_2012, Sillero_2013pof} and experimentally by  \cite{bailey_2013_JFM, Orlu_Schlatter_exp2013, marusic_2015}. Regarding near-equilibrium APG simulations, it is important to highlight the direct numerical simulation (DNS) by \cite{Kitsios2016} with a constant $\beta=1$; the self-similar DNS TBL at the verge of separation ($\beta=39$) by \cite{Kitsios2017} or the well-resolved large-eddy simulation (LES) database comprising different PG intensities by \cite{bobke2017}. 
Some relevant experimental databases include the near-equilibrium APG TBLs by \cite{skare_krogstad_1994, MTL_expSANMIGUEL}, and the studies by \cite{MONTY2011} and \cite{harun_monty_2013}, where the flow history was not controlled.

For a complete study of PG TBLs it is necessary to obtain databases with near-equilibrium conditions extending over long streamwise regions so the effects of the PG and the Reynolds number can be clearly identified and studied. 
In this study we contribute towards that goal with a new well-resolved LES of incompressible and near-equilibrium APG TBL over a flat-plate with a nearly-constant value of $\beta \simeq 1.4$ over a large Reynolds-number range up to $Re_{\theta} \simeq 8,700$ (where $Re_{\theta}$ is the Reynolds number defined in terms of edge velocity and displacement thickness). The $\beta$ value is not constant along the streamwise development of the TBL, but its rate of reduction is small enough to be considered as nearly-constant, since the APG effects are larger for low Reynolds numbers and here we focus on a region of high Reynolds number. A comparison with experiments at similarly high $\Rey$ and $\beta$ values is carried out in this work, and even if the flow history of the experimental $\beta$ exhibits a larger variation, the profiles of the simulation and the experiment are in good agreement, indicating that these small deviations from a constant $\beta$ are not relevant in the region of high Reynolds number.

This is one of the largest simulations of a near-equilibrium APG TBL extending over a Reynolds-number range comparable to that of wind-tunnel experiments \citep{MTL_expSANMIGUEL}. 
Throughout this work, the results will be compared with the well-resolved LES ZPG TBL by \cite{E-AmorZPG}, which exhibits a similar Reynolds-number range. These data-sets allow for a proper study of APG and Reynolds-number effects in the turbulent statistics as well as in the energetic scales involved in turbulence.

The article is organized as follows: in section \ref{sec:NumSetUp} the studied databases are presented, together with the numerical setup of the new simulation. 
The turbulent 2D statistics are compared with experimental data in section \ref{sec:RS_peaks_and_exp}.

Different scalings for the statistics will be considered in section \ref{sec:Scalings} and in order to understand the energetic scales as well as their distribution, the spectral analysis of the Reynolds stresses will be presented in section \ref{sec:Spectra}. 
In section \ref{sec:Conclusions} some conclusions on APG and Reynolds-number effects will be drawn, and an outlook will also be given. In appendix \hyperlink{AppA}{A} we include for the sake of completeness other turbulent statistics such as the streamwise evolution of integral parameters and the turbulent kinetic energy (TKE) budgets as well as the other 2D statistics in outer scaling that will serve as a support material to document the phenomena seen in the Reynolds stresses and spectra seen in previous sections.
Finally, appendix \hyperlink{AppB}{B} gives a breve description of the subgrid-scale model used in the present simulation.
