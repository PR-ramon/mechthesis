
\subsubsection{Interscale transport of energy.}

From the Reynolds transport equation or the KE, MKE and TKE budget equations, it was possible to obtain a term called ``Production'' which is composed by the RS and the mean velocity gradients, and it can be interpreted as a source of turbulent energy in the TKE equation and a sink in the MKE equation. Their contribution are cancelling each other when added to obtain the KE equation.

Following this idea, a similar term is looked for in the Reynolds-stress transport equations, that may indicate the transfer of energy between large scales and small scales.
In order to obtain such a decomposition, we follow the steps in \cite{PRL2018_Kawata, JFM2019_kawata_alfredsson}, where the perturbation velocity is further decomposed in a contribution from large scales $u_i^L$ and a contribution from small scales $u_i^S$, where the separation between large or small scales is made by a cutoff frequency/wavenumber in Fourier space $k_{z,c}$.
\begin{equation}
    \label{eq:intersc_decomp}
    u_i\myprime = u_i^L + u_i^S,
\end{equation}
\begin{equation}
    \label{eq:intersc_decomp}
    \langle u_i\myprime u_j\myprime \rangle_z = 
    \langle u_i^L u_j^L \rangle_z + 
    \langle u_i^S u_i^S \rangle_z
\end{equation}
\begin{equation}
    \label{eq:intersc_decomp}
    \overline{u_i\myprime u_j\myprime} = 
    \overline{ u_i^L u_j^L} + 
    \overline{ u_i^S u_i^S}
\end{equation}

\begin{equation}
    \label{eq:intersc_decomp}
    \overline{ u_i^L u_j^S} = 
    \overline{ u_i^S u_j^L} = 0
\end{equation}



\begin{itemize}
    \item 
\end{itemize}