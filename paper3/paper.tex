%------------------------------------------------------------------------------
% Define title, author(s), affiliation and publishing status
%
\papertitle[Skin-friction contributions in adverse-pressure-gradient turbulent boundary layers] % Short title used in headlines (optional)
{%
  A new perspective on skin-friction contributions in adverse-pressure-gradient turbulent boundary layers% THE COMMENT SYMBOL AT THE END OF THIS LINE IS NEEDED
}%
%
\papertoctitle{Skin-friction contributions in adverse-pressure-gradient \\ turbulent boundary layers} % Title for toc
%
% \paperauthor[M. Atzori, F. Mallor, R. Pozuelo, A. Stroh, D. Gatti, K. Fukagata, R. Vinuesa \& P. Schlatter] % Short authors used in headlines and List Of Papers
\paperauthor[M. Atzori {\it et al.}] % Short authors used in headlines and List Of Papers
{%
  Marco Atzori$^1$, Ferm\'in Mallor$^2$, Ram\'on Pozuelo$^2$, Alexander Stroh$^3$, Davide Gatti$^3$, Koji Fukagata$^4$, Ricardo Vinuesa$^2$ \& Philipp Schlatter$^2$%  
}%
%
%\listpaperauthor{A. Skywalker \& D. Vader}% (optional) Short authors used in List Of Papers
%
\paperaffiliation
{%
  $^1$ Department of Particulate Flow Modelling, Johannes Kepler University, 4040 Linz, Austria\\
  $^2$ FLOW, Engineering Mechanics, KTH Royal Institute of Technology, SE-100 44 Stockholm, Sweden\\
  $^3$ Institute of Fluid Mechanics (ISTM), Karlsruhe Institute of Technology, 76131 Karlsruhe, Germany \\
  $^4$ Department of Mechanical Engineering, Keio University, 223-8522 Yokohama, Japan %
}%
%
\paperjournal[Phys. Fluids] % Short publish info used in List Of Papers
{%
	Physics of Fluids%
}%
%
\papervolume{AAAAA}%
%
\papernumber{4}%
%
\paperpages{AAAAAA}%
%
\paperyear{2022}%
%
\papersummary%
{% Insert summary of the paper here (used in introduction) 
	BBBBBBBB
}%
%
% 
% 
\graphicspath{{paper3/figs}}%
%
%
%===============================================================================
%                            BEGIN PAPER
%===============================================================================
%
\begin{paper}

\makepapertitle

%------------------------------------------------------------------------------
% Abstract
%------------------------------------------------------------------------------
%
\begin{paperabstract}
    For adverse-pressure-gradient turbulent boundary layers, the study of integral skin-friction contributions still poses significant challenges. Beyond questions related to the integration boundaries and the derivation procedure, which have been thoroughly investigated in the literature, an important issue is how different terms should be aggregated. The nature of these flows, which exhibit significant in-homogeneity in the streamwise direction, usually results in cancellation between several contributions with high absolute values. We propose a formulation of the identity derived by Fukagata, Iwamoto \& Kasagi (Phys. Fluids, vol. 14, 2002, pp. 73--76), which we obtained from the convective form of the governing equations. A new skin-friction contribution is defined, considering wall-tangential convection and pressure gradient together. This contribution is related to the evolution of the dynamic pressure in the mean flow. The results of the decomposition are examined for a broad range of pressure-gradient conditions and different flow-control strategies. We found that the new formulation of the identity allows to readily identify the different regimes of near-equilibrium conditions and approaching separation. It also provides a more effective description of control effects. A similar aggregation between convection and pressure-gradient terms is also possible for any other decomposition where in-homogeneity contributions are considered explicitly. 
\end{paperabstract}


%------------------------------------------------------------------------------
% Article
%------------------------------------------------------------------------------
%
% \documentclass[aps,prl,reprint]{revtex4-2}

% \usepackage{graphicx}
\usepackage{epstopdf, epsfig}
\usepackage{amsmath}
\usepackage[title]{appendix}
% \usepackage[section]{placeins}
\usepackage{placeins}


% Hyperrefs
\usepackage{hyperref}
\hypersetup{
  colorlinks   = true, %Colours links instead of ugly boxes
  urlcolor     = blue, %Colour for external hyperlinks
  linkcolor    = blue, %Colour of internal links
  citecolor   = green %Colour of citations
}

% % Review Commands
\usepackage[normalem]{ulem}
\usepackage{color}
\newcommand{\rev}[1]{{\color{red}#1}}
\newcommand{\highlight}[1]{{\color{green}[#1]}}
\newcommand{\revd}[1]{{\color{blue}\sout{#1}}}
\newcommand{\revdd}[2]{\revd{#1} \rev{#2}}

\newtheorem{lemma}{Lemma}
\newtheorem{corollary}{Corollary}

\newcommand{\subf}[2]{%
  {\small\begin{tabular}[t]{@{}c@{}}
  #1\\#2
  \end{tabular}}%
}

% % Tikz commands
\usepackage{tikz}
\definecolor{magenta_1}{rgb}{0.64,0.08,0.18}
\definecolor{yellow_1}{rgb}{0.93,0.69,0.13}
% tikz commands for lines in captions
%\newcommand{\greenline}{\raisebox{2pt}{\tikz{\draw[-,black!40!green,solid,line width = 0.9pt](0,0) -- (5mm,0);}}}
\newcommand{\blueline}  {\raisebox{2pt}{\tikz{\draw[-,blue,solid,line width = 0.9pt](0,0) -- (5mm,0);}}}
\newcommand{\redline}   {\raisebox{2pt}{\tikz{\draw[-,red, solid,line width = 0.9pt](0,0) -- (5mm,0);}}}
\newcommand{\cyanline}  {\raisebox{2pt}{\tikz{\draw[-,cyan,solid,line width = 0.9pt](0,0) -- (5mm,0);}}}
\newcommand{\orangeline}{\raisebox{2pt}{\tikz{\draw[-,orange,solid,line width = 0.9pt](0,0) -- (5mm,0);}}}
\newcommand{\greenline} {\raisebox{2pt}{\tikz{\draw[-,green,solid,line width = 0.9pt](0,0) -- (5mm,0);}}}
\newcommand{\blackline} {\raisebox{2pt}{\tikz{\draw[-,black,solid,line width = 0.9pt](0,0) -- (5mm,0);}}}
\newcommand{\magentaline} {\raisebox{2pt}{\tikz{\draw[-,magenta_1,solid,line width = 0.9pt](0,0) -- (5mm,0);}}}
\newcommand{\yellowline} {\raisebox{2pt}{\tikz{\draw[-,yellow_1,solid,line width = 0.9pt](0,0) -- (5mm,0);}}}


\newcommand{\bluedash} {\raisebox{2pt}{\tikz{\draw[-,blue ,dashed,line width = 0.9pt](0,0) -- (5mm,0);}}}
\newcommand{\reddash}  {\raisebox{2pt}{\tikz{\draw[-,red  ,dashed,line width = 0.9pt](0,0) -- (5mm,0);}}}
\newcommand{\greendash}{\raisebox{2pt}{\tikz{\draw[-,green,dashed,line width = 0.9pt](0,0) -- (5mm,0);}}}
\newcommand{\blackdash}{\raisebox{2pt}{\tikz{\draw[-,black,dashed,line width = 0.9pt](0,0) -- (5mm,0);}}}
\newcommand{\magentadash}{\raisebox{2pt}{\tikz{\draw[-,magenta_1,dashed,line width = 0.9pt](0,0) -- (5mm,0);}}}

\newcommand{\blackcircle} {\raisebox{2pt}{\tikz{\draw [black]  (0,0) circle (2pt);}}}
\newcommand{\bluecircle} {\raisebox{2pt}{\tikz{\draw [blue]  (0,0) circle (2pt);}}}
\newcommand{\redcircle} {\raisebox{2pt}{\tikz{\draw [red]  (0,0) circle (2pt);}}}


\newcommand{\bluepoint} {\raisebox{2pt}{\tikz{\filldraw [blue]  (0,0) circle (2pt);}}}
\newcommand{\redpoint}  {\raisebox{2pt}{\tikz{\filldraw [red]   (0,0) circle (2pt);}}}
\newcommand{\greenpoint}{\raisebox{2pt}{\tikz{\filldraw [green] (0,0) circle (2pt);}}}
\newcommand{\blackpoint}{\raisebox{2pt}{\tikz{\filldraw [black] (0,0) circle (2pt);}}}

\newcommand{\blackdotted}{\raisebox{2pt}{\tikz{\draw[-,black ,dotted,line width = 0.9pt](0,0) -- (5mm,0);}}}
\newcommand{\blackdashdot}{\raisebox{2pt}{\tikz{\draw[-,black ,dashdotted,line width = 0.9pt](0,0) -- (5mm,0);}}}

\newcommand{\blackSquare} {\raisebox{2pt}{\tikz{\draw[black, very thick] (0,0) rectangle (3pt,3pt);}}}

\newcommand{\magentaDiamond} {\raisebox{2pt}{\tikz{\draw[magenta, very thick] (0,0) -- (2pt, 2pt) -- (4pt,0) -- (2pt,-2pt) -- cycle ;}}}



% \begin{document}

% \preprint{APS/123-QED}

% \title{New insight into the spectra of turbulent boundary layers with pressure gradients}
% \author{Ram\'on Pozuelo}
% \author{Qiang Li}%
% \author{Philipp Schlatter}
% \author{Ricardo Vinuesa}
% \affiliation{FLOW, Engineering Mechanics, KTH Royal Institute of Technology, SE-100 44 Stockholm, Sweden}


% \date{\today}

% % Abstracts:  ≤ 600 characters
% \begin{abstract}

% \end{abstract}
% % 248 words

% \maketitle

\section{Introduction}


Turbulent boundary layers (TBLs) play an essential role in a wide range of areas, including atmospheric flows, aircraft design or turbine blades. Assessing the effect of streamwise pressure gradients on TBLs is critical to completely understand these applications \cite{harun_monty2013, Maciel_2018}, and this is very challenging due to the complex effect of flow history on the local state of turbulence \cite{bobke2017, tanarro_2020, Kitsios2017}. Here we study adverse-pressure-gradient (APG) effects on TBLs on statistically two-dimensional flows subjected to a nearly-constant APG magnitude leading to near-equilibrium conditions \cite{Marusic_McKeon_2010, Nagib_2008}.
% The turbulent character of the flow is analysed through the Reynolds decomposition \cite{Rey_decomp} of the fluid variables into a mean flow (averaged in time and the homogeneous spanwise direction) and a fluctuating component.
% Through this statistical approach, the mean flow has been extensively studied and some universal laws have been established, such as the law of the wall \cite{vonKarman1931}, the defect law and log-law  \cite{millikan39}, together with various approaches to account for different flow configurations \cite{Luchini_2017}. 
% One important tool for mathematical modeling of physical systems is scaling, which in the context of turbulent flows has been used for assessing the physics at higher Reynolds numbers~\cite{Hultmark_2012}, and also in combination with symmetry analyses~\cite{Oberlack_2022}.
% The approach taken in this work is based on scaling different energetic regions of the boundary layer at high Reynolds numbers, and it can be useful for other multi-scale systems where there are different energy scales interacting with each other. In these cases, a marginal integration of the scales of various sizes can lead to understanding the relative relevance and sensitivity of the various scales.

% In this study we assess the fluctuating components of APG TBLs, denoted by Reynolds stresses (RSs), where the streamwise normal component $\overline{u^2}$ is the most energetic. Note that $\overline{(\cdot)}$ denotes the average in time and the homogeneous spanwise direction.

% In APG TBLs we deal mainly with two parameters: the Reynolds number and the APG magnitude, where the latter can also be a function of the streamwise position.
% % \rev{It is interesting to find scaling factors for certain characteristics of the flow such that they can collapse with Reynolds number and/or APG magnitude, so as to further understand the physical behavior of the flow.}
% Kitsios {\it et al.}~\cite{Kitsios2016, Kitsios2017} used an integral approach to self-similarity in order to obtain scaling factors for the Reynolds stresses, and compared the spectra of a ZPG and an APG. 
% The integral scaling factor they used was the displacement thickness, which is defined as $\delta^*=\int_0^{\delta_{99}} (1 - U/U_{e}) \mathrm{d}y $, where $U$ is the mean streamwise velocity and $U_e$ is the velocity at the edge of the boundary layer ($y=\delta_{99}$). Note that $x$, $y$ and $z$ are the streamwise, wall-normal and spanwise coordinates, respectively. 
% % \rev{The outer peak of all the RSs was shown to be at $y \approx 1.2\delta^*$ in Kitsios {\it et al.}~\cite{Kitsios2016} for their mild-APG simulation and at $y \approx \delta^*$ for their strong-APG case~\cite{Kitsios2017}. }
% Following this RS scaling, Maciel {\it et al.}~\cite{Maciel_2018} and Sanmiguel Vila {\it et al.}~\cite{Sanmiguel_PRF} scaled the outer peaks of the RSs with $\delta^*$ and $\delta_{99}$ for several numerical and experimental databases with a large range of APG magnitudes. 
% % \rev{ The former focused on non-equilibrium databases, such as the APG TBLs around wing profiles, and the latter showed results for experimental near-equilibrium TBLs developing on flat plates. }
% In these studies, it was found that the wall-normal location of the outer peak of the RSs scales with $\delta^*$, making it a scaling independent of the Reynolds number and APG effects. 

% Note that Maciel {\it et al.}~\cite{Maciel_2018} reflected on the $\delta^*$ scaling of the wall-normal location of the outer peak, $y_{\rm OP}$. Since $\delta^*/\delta_{99}$ decays slowly to zero when the Reynolds number tends to infinity, this would imply that the outer peak approaches the wall, therefore $\delta^*$ would only scale $y_{\rm OP}$, but not necessarily the size of the outer region.

% Despite the different values for the wall-normal location of the outer peaks in the RSs for moderate and strong APGs documented in Ref.~\cite{Kitsios2017}, the spectral outer peak for the ZPG and the strong-APG cases were shown to be at the same wall-normal location: $y \approx \delta^*$. 

% In other studies where $\delta_{99}$ was used for scaling the outer region of the TBL \cite{Lee2017, bobke2017}, the outer-spectral-peak wavelength $\lambda_z$ appears to scale better in both ZPG and APG with $\delta_{99}$ than with $\delta^*$. 

% Based on the work by Pozuelo {\it et al.}~\cite{Pozuelo_JFM_22}, in APG TBLs the wall-normal location and magnitude of the near-wall peak, denoted here by $y_{\rm IP}$ and $\overline{u^2}_{\rm IP}$ respectively, were found to increase with Reynolds number and APG magnitude using inner scaling. In this scaling, the viscous length $\ell_{\tau}=\nu/u_{\tau}$ and the friction velocity $u_{\tau}$ are employed, where $\nu$ is the fluid kinematic viscosity and $u_{\tau}=\sqrt{\tau_w/\rho}$ (where $\tau_w$ is the wall-shear stress and $\rho$ is the fluid density). The superscript `+'  will be used to denote inner scaling.
% Regarding the outer peak, its wall-normal location $y_{\rm OP}$ was found to be approximately constant when scaled with either the displacement thickness $\delta^*$ or the boundary-layer thickness $\delta_{99}$, since for a moderate APG $\delta^*/\delta_{99}$ decreases slowly with the Reynolds number.
% With the former length scale, $y_{\rm OP}/\delta^*$ appeared to be less influenced by the APG, obtaining an approximate value of 1.4, while using $\delta_{99}$, the locations varied with APG magnitude. 


% \section{Databases}
In this study we analyze two high-Reynolds-number databases of well-resolved LES of ZPG \cite{EAmorZPG} and APG TBLs \cite{Pozuelo_JFM_22}, focusing on the near-equilibrium region in the APG, {\it i.e.} for $Re_{\tau}>1000$. Note that $Re_{\tau}$ is the friction Reynolds number based on the friction velocity $u_{\tau}$, and can be written as $Re_{\tau}=\delta_{99}/\ell_{\tau}=\delta_{99}^+$.
The colours and line styles used for the two databases are shown in Table~ \ref{tab:PGcases}.
% ---------------------------------------------------
% ------------Databases and symbols------------------
% ---------------------------------------------------
\begin{table}
\centering
\begin{tabular}{lc||cr}
\textrm{PG } & \textrm{Colour} & \textrm{$Re_{\tau}$} & \textrm{Style} \\
% \colrule
APG & $\orangeline$ or $\lightorangeline$ & 1000 & $\blackline$  \\
ZPG & $\blackline$ or $\lightgrayline$  & 1500 & $\blackdash$  \\
    &                & 2000 & $\blackdotted$ \\
\end{tabular}
\caption{(Left) Colors used to denote the contour lines of the two TBL databases and (right) line styles used to denote the analyzed profiles at different friction Reynolds numbers in the two TBL cases. }
\label{tab:PGcases}
\end{table} 
% ---------------------------------------------------
% \section{Equations}
\section{\label{sec:Equations} Equations }
% ---------------------------------------------------

% Total velocities
\begin{equation}
    \textcolor{black}{\pdv{u_i}{t} + u_k\pdv{u_i}{x_k} = - \pdv{p}{x_i} + \nu \pdv[2]{u_i}{x_k}}
\end{equation}

% RANS
\begin{equation}
    \textcolor{mygreen2}{\pdv{U_i}{t} + U_k\pdv{U_i}{x_k} + \pdv{\overline{u_i u_k}}{x_k} = - \pdv{P}{x_k} + \nu \pdv[2]{U_i}{x_k}}
\end{equation}

% Perturbations
\begin{equation}
    \textcolor{myorange1}{\pdv{u_i\myprime}{t} + 
    U_k\pdv{u_i\myprime}{x_k} + u_k\myprime\pdv{u_i}{x_k} - \pdv{\overline{u_i\myprime u_k\myprime}}{x_k}  =
    - \pdv{p\myprime}{x_i} + \nu \pdv[2]{u_k\myprime}{x_k}}
\end{equation}

% Frequency cutting
\begin{equation}
    u_i\myprime = u_i^L + u_i^S
\end{equation}
\input{paper5/Results}
\input{paper5/Conclusion}

%Acknowledgements
RV acknowledges the financial support provided by the Swedish Research Council (VR).
PS and QL were supported by the KAW Foundatation.
The computations and data handling were enabled by resources provided by the Swedish National Infrastructure for Computing (SNIC), partially funded by the Swedish Research Council.

% \end{document}
% \bibliography{apssamp}% Produces the bibliography via BibTeX.

% \end{document}
%
% ****** End of file apssamp.tex ******


%------------------------------------------------------------------------------
% Bibliography
%------------------------------------------------------------------------------
%
%\clearpage
\bibliographystyle{jfm}
\bibliography{thesis}
%
% \IfFileExists{paper2/paper.bbl}{%------------------------------------------------------------------------------
% Define title, author(s), affiliation and publishing status
%
\papertitle[Interscale energy transport in APG] % Short title used in headlines (optional)
{%
  Analysis of interscale energy transport in adverse pressure gradient turbulent boundary layers% THE COMMENT SYMBOL AT THE END OF THIS LINE IS NEEDED
}%
%
\papertoctitle{Analysis of interscale energy transport  \\ in adverse pressure gradient turbulent boundary layers} % Title for toc
%
\paperauthor[R. Pozuelo, A. Tanarro, P. Schlatter \& R.Vinuesa] % Short authors used in headlines and List Of Papers
{%
  Ram\'on Pozuelo$^1$, \'Alvaro Tanarro$^1$, Philipp Schlatter$^1$ and Ricardo Vinuesa$^1$%
}%
%
%\listpaperauthor{A. Skywalker \& D. Vader}% (optional) Short authors used in List Of Papers
%
\paperaffiliation
{%
  $^1$ FLOW, Engineering Mechanics, KTH Royal Institute of Technology, SE-100 44 Stockholm, Sweden
}%
%
\paperjournal[J.~Fluid Mech.] % Short publish info used in List Of Papers
{%
	Journal of Fluid Mechanics%
}%
%
\papervolume{A5}%
%
\papernumber{B5}%
%
\paperpages{C5}%
%
\paperyear{2022}%
%
\papersummary%
{% Insert summary of the paper here (used in introduction) 
	D5
}%
% 
\graphicspath{{paper5/figs}}%
%
%
%===============================================================================
%                            BEGIN PAPER
%===============================================================================
%
\begin{paper}

\makepapertitle

%------------------------------------------------------------------------------
% Abstract
%------------------------------------------------------------------------------
%
\begin{paperabstract}
	Abstract of the interscale paper
\end{paperabstract}


%------------------------------------------------------------------------------
% Article
%------------------------------------------------------------------------------
%
% \documentclass[aps,prl,reprint]{revtex4-2}

% \usepackage{graphicx}
\usepackage{epstopdf, epsfig}
\usepackage{amsmath}
\usepackage[title]{appendix}
% \usepackage[section]{placeins}
\usepackage{placeins}


% Hyperrefs
\usepackage{hyperref}
\hypersetup{
  colorlinks   = true, %Colours links instead of ugly boxes
  urlcolor     = blue, %Colour for external hyperlinks
  linkcolor    = blue, %Colour of internal links
  citecolor   = green %Colour of citations
}

% % Review Commands
\usepackage[normalem]{ulem}
\usepackage{color}
\newcommand{\rev}[1]{{\color{red}#1}}
\newcommand{\highlight}[1]{{\color{green}[#1]}}
\newcommand{\revd}[1]{{\color{blue}\sout{#1}}}
\newcommand{\revdd}[2]{\revd{#1} \rev{#2}}

\newtheorem{lemma}{Lemma}
\newtheorem{corollary}{Corollary}

\newcommand{\subf}[2]{%
  {\small\begin{tabular}[t]{@{}c@{}}
  #1\\#2
  \end{tabular}}%
}

% % Tikz commands
\usepackage{tikz}
\definecolor{magenta_1}{rgb}{0.64,0.08,0.18}
\definecolor{yellow_1}{rgb}{0.93,0.69,0.13}
% tikz commands for lines in captions
%\newcommand{\greenline}{\raisebox{2pt}{\tikz{\draw[-,black!40!green,solid,line width = 0.9pt](0,0) -- (5mm,0);}}}
\newcommand{\blueline}  {\raisebox{2pt}{\tikz{\draw[-,blue,solid,line width = 0.9pt](0,0) -- (5mm,0);}}}
\newcommand{\redline}   {\raisebox{2pt}{\tikz{\draw[-,red, solid,line width = 0.9pt](0,0) -- (5mm,0);}}}
\newcommand{\cyanline}  {\raisebox{2pt}{\tikz{\draw[-,cyan,solid,line width = 0.9pt](0,0) -- (5mm,0);}}}
\newcommand{\orangeline}{\raisebox{2pt}{\tikz{\draw[-,orange,solid,line width = 0.9pt](0,0) -- (5mm,0);}}}
\newcommand{\greenline} {\raisebox{2pt}{\tikz{\draw[-,green,solid,line width = 0.9pt](0,0) -- (5mm,0);}}}
\newcommand{\blackline} {\raisebox{2pt}{\tikz{\draw[-,black,solid,line width = 0.9pt](0,0) -- (5mm,0);}}}
\newcommand{\magentaline} {\raisebox{2pt}{\tikz{\draw[-,magenta_1,solid,line width = 0.9pt](0,0) -- (5mm,0);}}}
\newcommand{\yellowline} {\raisebox{2pt}{\tikz{\draw[-,yellow_1,solid,line width = 0.9pt](0,0) -- (5mm,0);}}}


\newcommand{\bluedash} {\raisebox{2pt}{\tikz{\draw[-,blue ,dashed,line width = 0.9pt](0,0) -- (5mm,0);}}}
\newcommand{\reddash}  {\raisebox{2pt}{\tikz{\draw[-,red  ,dashed,line width = 0.9pt](0,0) -- (5mm,0);}}}
\newcommand{\greendash}{\raisebox{2pt}{\tikz{\draw[-,green,dashed,line width = 0.9pt](0,0) -- (5mm,0);}}}
\newcommand{\blackdash}{\raisebox{2pt}{\tikz{\draw[-,black,dashed,line width = 0.9pt](0,0) -- (5mm,0);}}}
\newcommand{\magentadash}{\raisebox{2pt}{\tikz{\draw[-,magenta_1,dashed,line width = 0.9pt](0,0) -- (5mm,0);}}}

\newcommand{\blackcircle} {\raisebox{2pt}{\tikz{\draw [black]  (0,0) circle (2pt);}}}
\newcommand{\bluecircle} {\raisebox{2pt}{\tikz{\draw [blue]  (0,0) circle (2pt);}}}
\newcommand{\redcircle} {\raisebox{2pt}{\tikz{\draw [red]  (0,0) circle (2pt);}}}


\newcommand{\bluepoint} {\raisebox{2pt}{\tikz{\filldraw [blue]  (0,0) circle (2pt);}}}
\newcommand{\redpoint}  {\raisebox{2pt}{\tikz{\filldraw [red]   (0,0) circle (2pt);}}}
\newcommand{\greenpoint}{\raisebox{2pt}{\tikz{\filldraw [green] (0,0) circle (2pt);}}}
\newcommand{\blackpoint}{\raisebox{2pt}{\tikz{\filldraw [black] (0,0) circle (2pt);}}}

\newcommand{\blackdotted}{\raisebox{2pt}{\tikz{\draw[-,black ,dotted,line width = 0.9pt](0,0) -- (5mm,0);}}}
\newcommand{\blackdashdot}{\raisebox{2pt}{\tikz{\draw[-,black ,dashdotted,line width = 0.9pt](0,0) -- (5mm,0);}}}

\newcommand{\blackSquare} {\raisebox{2pt}{\tikz{\draw[black, very thick] (0,0) rectangle (3pt,3pt);}}}

\newcommand{\magentaDiamond} {\raisebox{2pt}{\tikz{\draw[magenta, very thick] (0,0) -- (2pt, 2pt) -- (4pt,0) -- (2pt,-2pt) -- cycle ;}}}



% \begin{document}

% \preprint{APS/123-QED}

% \title{New insight into the spectra of turbulent boundary layers with pressure gradients}
% \author{Ram\'on Pozuelo}
% \author{Qiang Li}%
% \author{Philipp Schlatter}
% \author{Ricardo Vinuesa}
% \affiliation{FLOW, Engineering Mechanics, KTH Royal Institute of Technology, SE-100 44 Stockholm, Sweden}


% \date{\today}

% % Abstracts:  ≤ 600 characters
% \begin{abstract}

% \end{abstract}
% % 248 words

% \maketitle

\section{Introduction}


Turbulent boundary layers (TBLs) play an essential role in a wide range of areas, including atmospheric flows, aircraft design or turbine blades. Assessing the effect of streamwise pressure gradients on TBLs is critical to completely understand these applications \cite{harun_monty2013, Maciel_2018}, and this is very challenging due to the complex effect of flow history on the local state of turbulence \cite{bobke2017, tanarro_2020, Kitsios2017}. Here we study adverse-pressure-gradient (APG) effects on TBLs on statistically two-dimensional flows subjected to a nearly-constant APG magnitude leading to near-equilibrium conditions \cite{Marusic_McKeon_2010, Nagib_2008}.
% The turbulent character of the flow is analysed through the Reynolds decomposition \cite{Rey_decomp} of the fluid variables into a mean flow (averaged in time and the homogeneous spanwise direction) and a fluctuating component.
% Through this statistical approach, the mean flow has been extensively studied and some universal laws have been established, such as the law of the wall \cite{vonKarman1931}, the defect law and log-law  \cite{millikan39}, together with various approaches to account for different flow configurations \cite{Luchini_2017}. 
% One important tool for mathematical modeling of physical systems is scaling, which in the context of turbulent flows has been used for assessing the physics at higher Reynolds numbers~\cite{Hultmark_2012}, and also in combination with symmetry analyses~\cite{Oberlack_2022}.
% The approach taken in this work is based on scaling different energetic regions of the boundary layer at high Reynolds numbers, and it can be useful for other multi-scale systems where there are different energy scales interacting with each other. In these cases, a marginal integration of the scales of various sizes can lead to understanding the relative relevance and sensitivity of the various scales.

% In this study we assess the fluctuating components of APG TBLs, denoted by Reynolds stresses (RSs), where the streamwise normal component $\overline{u^2}$ is the most energetic. Note that $\overline{(\cdot)}$ denotes the average in time and the homogeneous spanwise direction.

% In APG TBLs we deal mainly with two parameters: the Reynolds number and the APG magnitude, where the latter can also be a function of the streamwise position.
% % \rev{It is interesting to find scaling factors for certain characteristics of the flow such that they can collapse with Reynolds number and/or APG magnitude, so as to further understand the physical behavior of the flow.}
% Kitsios {\it et al.}~\cite{Kitsios2016, Kitsios2017} used an integral approach to self-similarity in order to obtain scaling factors for the Reynolds stresses, and compared the spectra of a ZPG and an APG. 
% The integral scaling factor they used was the displacement thickness, which is defined as $\delta^*=\int_0^{\delta_{99}} (1 - U/U_{e}) \mathrm{d}y $, where $U$ is the mean streamwise velocity and $U_e$ is the velocity at the edge of the boundary layer ($y=\delta_{99}$). Note that $x$, $y$ and $z$ are the streamwise, wall-normal and spanwise coordinates, respectively. 
% % \rev{The outer peak of all the RSs was shown to be at $y \approx 1.2\delta^*$ in Kitsios {\it et al.}~\cite{Kitsios2016} for their mild-APG simulation and at $y \approx \delta^*$ for their strong-APG case~\cite{Kitsios2017}. }
% Following this RS scaling, Maciel {\it et al.}~\cite{Maciel_2018} and Sanmiguel Vila {\it et al.}~\cite{Sanmiguel_PRF} scaled the outer peaks of the RSs with $\delta^*$ and $\delta_{99}$ for several numerical and experimental databases with a large range of APG magnitudes. 
% % \rev{ The former focused on non-equilibrium databases, such as the APG TBLs around wing profiles, and the latter showed results for experimental near-equilibrium TBLs developing on flat plates. }
% In these studies, it was found that the wall-normal location of the outer peak of the RSs scales with $\delta^*$, making it a scaling independent of the Reynolds number and APG effects. 

% Note that Maciel {\it et al.}~\cite{Maciel_2018} reflected on the $\delta^*$ scaling of the wall-normal location of the outer peak, $y_{\rm OP}$. Since $\delta^*/\delta_{99}$ decays slowly to zero when the Reynolds number tends to infinity, this would imply that the outer peak approaches the wall, therefore $\delta^*$ would only scale $y_{\rm OP}$, but not necessarily the size of the outer region.

% Despite the different values for the wall-normal location of the outer peaks in the RSs for moderate and strong APGs documented in Ref.~\cite{Kitsios2017}, the spectral outer peak for the ZPG and the strong-APG cases were shown to be at the same wall-normal location: $y \approx \delta^*$. 

% In other studies where $\delta_{99}$ was used for scaling the outer region of the TBL \cite{Lee2017, bobke2017}, the outer-spectral-peak wavelength $\lambda_z$ appears to scale better in both ZPG and APG with $\delta_{99}$ than with $\delta^*$. 

% Based on the work by Pozuelo {\it et al.}~\cite{Pozuelo_JFM_22}, in APG TBLs the wall-normal location and magnitude of the near-wall peak, denoted here by $y_{\rm IP}$ and $\overline{u^2}_{\rm IP}$ respectively, were found to increase with Reynolds number and APG magnitude using inner scaling. In this scaling, the viscous length $\ell_{\tau}=\nu/u_{\tau}$ and the friction velocity $u_{\tau}$ are employed, where $\nu$ is the fluid kinematic viscosity and $u_{\tau}=\sqrt{\tau_w/\rho}$ (where $\tau_w$ is the wall-shear stress and $\rho$ is the fluid density). The superscript `+'  will be used to denote inner scaling.
% Regarding the outer peak, its wall-normal location $y_{\rm OP}$ was found to be approximately constant when scaled with either the displacement thickness $\delta^*$ or the boundary-layer thickness $\delta_{99}$, since for a moderate APG $\delta^*/\delta_{99}$ decreases slowly with the Reynolds number.
% With the former length scale, $y_{\rm OP}/\delta^*$ appeared to be less influenced by the APG, obtaining an approximate value of 1.4, while using $\delta_{99}$, the locations varied with APG magnitude. 


% \section{Databases}
In this study we analyze two high-Reynolds-number databases of well-resolved LES of ZPG \cite{EAmorZPG} and APG TBLs \cite{Pozuelo_JFM_22}, focusing on the near-equilibrium region in the APG, {\it i.e.} for $Re_{\tau}>1000$. Note that $Re_{\tau}$ is the friction Reynolds number based on the friction velocity $u_{\tau}$, and can be written as $Re_{\tau}=\delta_{99}/\ell_{\tau}=\delta_{99}^+$.
The colours and line styles used for the two databases are shown in Table~ \ref{tab:PGcases}.
% ---------------------------------------------------
% ------------Databases and symbols------------------
% ---------------------------------------------------
\begin{table}
\centering
\begin{tabular}{lc||cr}
\textrm{PG } & \textrm{Colour} & \textrm{$Re_{\tau}$} & \textrm{Style} \\
% \colrule
APG & $\orangeline$ or $\lightorangeline$ & 1000 & $\blackline$  \\
ZPG & $\blackline$ or $\lightgrayline$  & 1500 & $\blackdash$  \\
    &                & 2000 & $\blackdotted$ \\
\end{tabular}
\caption{(Left) Colors used to denote the contour lines of the two TBL databases and (right) line styles used to denote the analyzed profiles at different friction Reynolds numbers in the two TBL cases. }
\label{tab:PGcases}
\end{table} 
% ---------------------------------------------------
% \section{Equations}
\section{\label{sec:Equations} Equations }
% ---------------------------------------------------

% Total velocities
\begin{equation}
    \textcolor{black}{\pdv{u_i}{t} + u_k\pdv{u_i}{x_k} = - \pdv{p}{x_i} + \nu \pdv[2]{u_i}{x_k}}
\end{equation}

% RANS
\begin{equation}
    \textcolor{mygreen2}{\pdv{U_i}{t} + U_k\pdv{U_i}{x_k} + \pdv{\overline{u_i u_k}}{x_k} = - \pdv{P}{x_k} + \nu \pdv[2]{U_i}{x_k}}
\end{equation}

% Perturbations
\begin{equation}
    \textcolor{myorange1}{\pdv{u_i\myprime}{t} + 
    U_k\pdv{u_i\myprime}{x_k} + u_k\myprime\pdv{u_i}{x_k} - \pdv{\overline{u_i\myprime u_k\myprime}}{x_k}  =
    - \pdv{p\myprime}{x_i} + \nu \pdv[2]{u_k\myprime}{x_k}}
\end{equation}

% Frequency cutting
\begin{equation}
    u_i\myprime = u_i^L + u_i^S
\end{equation}
\input{paper5/Results}
\input{paper5/Conclusion}

%Acknowledgements
RV acknowledges the financial support provided by the Swedish Research Council (VR).
PS and QL were supported by the KAW Foundatation.
The computations and data handling were enabled by resources provided by the Swedish National Infrastructure for Computing (SNIC), partially funded by the Swedish Research Council.

% \end{document}
% \bibliography{apssamp}% Produces the bibliography via BibTeX.

% \end{document}
%
% ****** End of file apssamp.tex ******


%------------------------------------------------------------------------------
% Bibliography
%------------------------------------------------------------------------------
%
%\clearpage
\bibliographystyle{jfm}
\bibliography{thesis}
%
% \IfFileExists{paper2/paper.bbl}{%------------------------------------------------------------------------------
% Define title, author(s), affiliation and publishing status
%
\papertitle[Interscale energy transport in APG] % Short title used in headlines (optional)
{%
  Analysis of interscale energy transport in adverse pressure gradient turbulent boundary layers% THE COMMENT SYMBOL AT THE END OF THIS LINE IS NEEDED
}%
%
\papertoctitle{Analysis of interscale energy transport  \\ in adverse pressure gradient turbulent boundary layers} % Title for toc
%
\paperauthor[R. Pozuelo, A. Tanarro, P. Schlatter \& R.Vinuesa] % Short authors used in headlines and List Of Papers
{%
  Ram\'on Pozuelo$^1$, \'Alvaro Tanarro$^1$, Philipp Schlatter$^1$ and Ricardo Vinuesa$^1$%
}%
%
%\listpaperauthor{A. Skywalker \& D. Vader}% (optional) Short authors used in List Of Papers
%
\paperaffiliation
{%
  $^1$ FLOW, Engineering Mechanics, KTH Royal Institute of Technology, SE-100 44 Stockholm, Sweden
}%
%
\paperjournal[J.~Fluid Mech.] % Short publish info used in List Of Papers
{%
	Journal of Fluid Mechanics%
}%
%
\papervolume{A5}%
%
\papernumber{B5}%
%
\paperpages{C5}%
%
\paperyear{2022}%
%
\papersummary%
{% Insert summary of the paper here (used in introduction) 
	D5
}%
% 
\graphicspath{{paper5/figs}}%
%
%
%===============================================================================
%                            BEGIN PAPER
%===============================================================================
%
\begin{paper}

\makepapertitle

%------------------------------------------------------------------------------
% Abstract
%------------------------------------------------------------------------------
%
\begin{paperabstract}
	Abstract of the interscale paper
\end{paperabstract}


%------------------------------------------------------------------------------
% Article
%------------------------------------------------------------------------------
%
% \documentclass[aps,prl,reprint]{revtex4-2}

% \input{preamble}


% \begin{document}

% \preprint{APS/123-QED}

% \title{New insight into the spectra of turbulent boundary layers with pressure gradients}
% \author{Ram\'on Pozuelo}
% \author{Qiang Li}%
% \author{Philipp Schlatter}
% \author{Ricardo Vinuesa}
% \affiliation{FLOW, Engineering Mechanics, KTH Royal Institute of Technology, SE-100 44 Stockholm, Sweden}


% \date{\today}

% % Abstracts:  ≤ 600 characters
% \begin{abstract}

% \end{abstract}
% % 248 words

% \maketitle

\input{paper5/Intro}
\input{paper5/Database}
\input{paper5/Equations}
\input{paper5/Results}
\input{paper5/Conclusion}

%Acknowledgements
RV acknowledges the financial support provided by the Swedish Research Council (VR).
PS and QL were supported by the KAW Foundatation.
The computations and data handling were enabled by resources provided by the Swedish National Infrastructure for Computing (SNIC), partially funded by the Swedish Research Council.

% \end{document}
% \bibliography{apssamp}% Produces the bibliography via BibTeX.

% \end{document}
%
% ****** End of file apssamp.tex ******


%------------------------------------------------------------------------------
% Bibliography
%------------------------------------------------------------------------------
%
%\clearpage
\bibliographystyle{jfm}
\bibliography{thesis}
%
% \IfFileExists{paper2/paper.bbl}{%------------------------------------------------------------------------------
% Define title, author(s), affiliation and publishing status
%
\papertitle[Interscale energy transport in APG] % Short title used in headlines (optional)
{%
  Analysis of interscale energy transport in adverse pressure gradient turbulent boundary layers% THE COMMENT SYMBOL AT THE END OF THIS LINE IS NEEDED
}%
%
\papertoctitle{Analysis of interscale energy transport  \\ in adverse pressure gradient turbulent boundary layers} % Title for toc
%
\paperauthor[R. Pozuelo, A. Tanarro, P. Schlatter \& R.Vinuesa] % Short authors used in headlines and List Of Papers
{%
  Ram\'on Pozuelo$^1$, \'Alvaro Tanarro$^1$, Philipp Schlatter$^1$ and Ricardo Vinuesa$^1$%
}%
%
%\listpaperauthor{A. Skywalker \& D. Vader}% (optional) Short authors used in List Of Papers
%
\paperaffiliation
{%
  $^1$ FLOW, Engineering Mechanics, KTH Royal Institute of Technology, SE-100 44 Stockholm, Sweden
}%
%
\paperjournal[J.~Fluid Mech.] % Short publish info used in List Of Papers
{%
	Journal of Fluid Mechanics%
}%
%
\papervolume{A5}%
%
\papernumber{B5}%
%
\paperpages{C5}%
%
\paperyear{2022}%
%
\papersummary%
{% Insert summary of the paper here (used in introduction) 
	D5
}%
% 
\graphicspath{{paper5/figs}}%
%
%
%===============================================================================
%                            BEGIN PAPER
%===============================================================================
%
\begin{paper}

\makepapertitle

%------------------------------------------------------------------------------
% Abstract
%------------------------------------------------------------------------------
%
\begin{paperabstract}
	Abstract of the interscale paper
\end{paperabstract}


%------------------------------------------------------------------------------
% Article
%------------------------------------------------------------------------------
%
\input{paper5/article.tex}


%------------------------------------------------------------------------------
% Bibliography
%------------------------------------------------------------------------------
%
%\clearpage
\bibliographystyle{jfm}
\bibliography{thesis}
%
% \IfFileExists{paper2/paper.bbl}{\input{paper2/paper.bbl}}{}

%===============================================================================
%                            END PAPER
%===============================================================================
\end{paper}
}{}

%===============================================================================
%                            END PAPER
%===============================================================================
\end{paper}
}{}

%===============================================================================
%                            END PAPER
%===============================================================================
\end{paper}
}{}

%===============================================================================
%                            END PAPER
%===============================================================================
\end{paper}

