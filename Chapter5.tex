% \section{SIMSON}

% \section{Equations}



% \section{MPI}


\section{LES}


\section{BCs}




Experimental databases are of great utility since what we measure comes from a realistic physical scenario, although there can always be errors due to the instruments used to measure or the possible interference of the tool with the flow.
The time resolution of the experiments depends on the frequency used by the measuring tool to obtain data, but the time span of the data is usually larger than the one we obtain in simulations.
Another advantage of experiments is the large Reynolds flows that can be studied.
A drawback of experimental measures is the limited space resolution and quantities that can be measured.

Numerical experiments precise of a good mathematical model of the physics involved and boundary conditions surrounding the studied domain. The model that we will use to simulate the flow in a boundary layer is given by the continuity and momentum equations Eqs.~(\ref{eq:continuity_cap2}) and (\ref{eq:NS_cap2}).

Once the space-time evolution model is set up, the discretization of the equations can also suppose a source of errors. After the equations are discretized, they have to be compiled into a program language that has to deal with the numerical computations and the memory management, often involving the use of large clusters where chunks of data has to be passed from one machine to another. The discrete computations, methods, and memory consumption can also introduce numerical errors and the code should also be optimized to be able to produce results in an acceptable time.
Once the main code is ready, several in/out routines should be implemented to be able to obtain the measurements required. IO operations also requires a large memory available and an increase in times of execution.
In our numerical simulations we have used the already developed code SIMSON which is an in-house KTH code that contains all the tools to extract statistics, timeseries, correlations and spectral decompositions of different quantities that we required for our studies.

\section{SIMSON code}
SIMSON stands for ``pseudo-\textbf{S}pectral solver for
\textbf{I}nco\textbf{M}pre\textbf{S}sible b\textbf{O}u\textbf{N}dary layer flows''. See \cite{simson_techrep} for a detailed information on the equations and implementation of the code.
It has been widely used for channel and boundary layer flows, the majority of it is written in Fortran 77 and its results of direct numerical simulation (DNS) and large-eddy simulation (LES) have been validated with numerous experimental databases.
The directions $x$ and $z$ are periodic directions and are discretized equispaced to be able to work in the Fourier space where derivatives reduce to the multiplication of complex numbers.
The wall-normal direction $y$ uses Chebyshev polynomials, ideal for channel flow since it accumulates many points close to the lower and upper walls. 




\section{Numerical setup} \label{sec:NumSetUp}
Instead of Eqs.~(\ref{eq:continuity_cap2}) and (\ref{eq:NS_cap2}), where the variables are the 3 components of the velocity and the pressure, these equations can be simplified into the 
velocity-vorticity formulation. In this formulation the effects of the continuity equation are already included in the evolution equation of the Laplacian of the wall-normal velocity ($\nabla^2 v$) and the wall-normal vorticity $\omega_y$. The pressure is not a variable in these two equations. The velocity $v$ can be obtained directly from $\nabla^2 v$ and the other components $u$ and $w$ can be obtained from the continuity equation and the definition of vorticity.
The pressure term reduces to solving a Laplace equation,
\begin{equation}
    \nabla^2 p = \pdv{H_i}{x_i} - \nabla^2 (\frac{1}{2} u_j u_j)
\end{equation}
where
\begin{equation}
    \pmb{H} = \pmb{u} \land \pmb{\omega },
\end{equation}
% \begin{equation}
%     \vec{H} = \vec{u} \times \vec{\omega }
% \end{equation}
and $\land$ is the vectorial or cross-product of two vectors. 
% $\times$ 

With this formulation only 2 equations with a similar form have to be solved (see \cite{simson_techrep}) and the other velocities can be calculated in a simpler way.


\section{Parallelization}
The spatial resolutions and size of the domains required for high Reynolds flows is challenging for numerical experiments, the amount of memory needed is only manageable if several machines or CPUs work as a cluster, performing computations on local chunks of the domain and sending the necessary information to other CPUs in the cluster. 
This passage of information is handled by parallelizing the code using  MPI (message passing interface) which allows for a parallel scalability able to run large simulations in an optimal time.
For different strategies of parallelization in SIMSON, see \cite{MPI_SIMSON_Qiang}, where a 1D and a 2D parallelization are explained. 
The option we used for our large $\Rey$ simulations was the 2D parallelization.
The time used by MPI can be said to be a time where the code is not productive, therefore a strategy based in another spectral code LISO \citep{Liso_Hoyas_PRL_2006} was implemented. The variables used by SIMSON are of kind double, however LISO operates with both double and single precision. To save time in the MPI fase, the data is casted and passed as single precission streams, and received in single precision and casted again into double precision.
This additional casting involves an overhead, however, when the amount of information passed is huge, the casting time is minimal compared to the reduction by half of the bytes involved in the communication.
For the current b1.4 simulation, 4096 cores where used, 64 in each Fourier direction.



\subsection{Set up of a large APG simulation}

To obtain a high-Reynolds-number simulation we take as a baseline the ZPG case by \cite{EAmorZPG}, which is an LES and was validated through DNS data by \cite{schlatter_orlu_2010}.
The box size is $(L_x, L_y , Lz)$, and we chose the same streamwise size $L_x$ as the ZPG case. The effects of the moderate APG conditions will cause an increase of the boundary layer thickness compared to ZPG conditions as well as the influence of wider scales, for that reason we doubled the size in the wall-normal ($L_y$) and spanwise directions ($L_z$) compared to the ZPG case. 
\highlight{\ref{tab:param}}
In order to have periodic conditions in the streamwise direction, all the flow that exits the domain has to enter again before the end of the domain. To recover the inflow conditions at the end of the domain, a fringe region imposes the necessary forces to recover the laminar flow conditions at the inlet.

The growth of the boundary layer implies a wall-normal convection that is increased by the adverse-pressure gradient, leading to large quantities of flow that has to enter through the top boundary under fringe conditions.
If the fringe region is short, the wall-normal velocities of the flow that enters through the fringe region will be so high that the Courant–Friedrichs–Lewy (CFL) condition is not fulfilled and the simulation becomes unstable.
Large fringe regions imply a smaller effective domain.
The accumulation of points in the top of the domain also influences the CFL condition. Different fringe lengths where tested, increasing progressively from the length used for the ZPG simulation until a satisfactory fringe length was found.

\section{Parameters of the simulation} \label{subsec:Param_sim}
The simulation starts from a laminar Blasius BL with a displacement thickness $\delta^*_0$ such that $Re_{\delta_0^*}=450$. The flow develops under ZPG conditions and turbulence is forced through a volume forcing \citep{schlatter_orlu_2012} at a streamwise position $x/\delta^*_0=10$.
Once we obtain a fully-turbulent boundary layer with low resolution we increase progressively the resolution and impose the APG boundary condition (BC) until we reach the desired resolution and APG BC.

\section{Boundary conditions} \label{subsec:BC}

In the Fourier directions, the boundary condition imposed is periodicity, in the streamwise direction the fringe region is necessary to assure periodicity by damping the turbulent fluctuations and reducing the boundary layer thickness up to the levels in the inflow.
The top boundary condition (free-flow) is responsible of the APG conditions seen by the boundary layer.
Since the laminar inflow is more sensitive to APGs, an initial ZPG is imposed and the APG conditions start at $x/\delta^*_0 = 350$ where the flow is already turbulent.
The APG imposed is such that near-equilibrium is obtained.

\subsection{Free-stream boundary conditions}
The three conditions imposed at the top of the domain are:

\begin{equation} 
    \frac{\partial W}{\partial y}=0,
\label{eq:BC_dWdy}
\end{equation}

\begin{equation} 
    \Omega_z = \frac{\partial V}{\partial x} - \frac{\partial U}{\partial y}=0,
\label{eq:BC_Vort_z}
\end{equation}

\begin{equation}
    % U(x) = U_{\rm ref}\left( 1 + \frac{x - x_{zpg}}{x_{b}} \right)^{-m}. 
    U_{\rm top}(x)= \left\{ \begin{array}{ll}
             U_{\rm ref}, &  x \leq x_z \\
             \\ U_{\rm ref}\left( 1 + \frac{x - x_{z}}{x_{b}} \right)^{-m}, & x > x_z,
             \end{array}
   \right.
\label{eq:BC_APG_U}
\end{equation}

% the first one relates to the statistical homogeneity in $z$, through the derivative in the wall-normal direction of the spanwise mean velocity as shown in (\ref{eq:BC_dWdy}); 
% the second BC is zero spanwise vorticity, given by (\ref{eq:BC_Vort_z}) as in \cite{Kitsios2016, Abe_2019};
% finally, the third BC will introduce the adverse pressure gradient through a decaying mean streamwise velocity $U_{\rm top}(x)$ using a power-law distribution. First, a constant velocity is used to provide an initial ZPG development and at $x=x_z=350$ the power law is applied as indicated in (\ref{eq:BC_APG_U}). At the end of the domain the fringe raises the velocity to the initial $U_{\rm ref}$ values. The power law was also used as in \cite{Bobke_2016}, based on the theoretical studies by \cite{Townsend_1956_structure} and \cite{mellor_gibson_1966} in order to obtain a near-equilibrium TBL.
% The use of (\ref{eq:BC_APG_U}) together with a fringe region may produce instabilities in the code where the fringe region starts, which were damped by the LES filter. The parameters of the power law were chosen as in the m16 simulation \citep{Bobke_2016}, {\it i.e.},  $x_z=350$,  $x_b=60$ and $m=0.16$.




