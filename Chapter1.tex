% Throughout my academic career, I have enjoyed the step-by-step process of learning and I have discovered a passion for sharing my knowledge with friends, colleagues and some of my students.
% In the same way the process of learning is unique, the process of teaching or communicating the knowledge should also be personal, trying to achieve the main goal of communication, transmitting correct information.
% When we try to learn about a subject we can go to books, classes, different teachers, publications..., and sometimes we understand better from a book, other times from a video or a specific teacher or friend.
% In this thesis, I will present some results on the physics of turbulent fluid mechanics where I have been involved, and I will try my best to communicate this knowledge to those interested in reading this work even if the reader is not very familiar with fluid dynamics.
% Ideally, this transmission of knowledge should be concise, but I have experienced that certain complex knowledge sometimes requires different points of view or a simpler language. Personally, I like to communicate my knowledge as if it was a conversation instead of formal and impersonal writing.

Fluid mechanics 
it really is a complex science, but we already have intuitive and practical knowledge since we live and manipulate fluids every day, so here we will
explain the basics.
In physics, a fluid is a liquid, gas, or other material that continuously deforms (flows) under an applied shear stress, or external force.
A fluid is a substance that \highlight{can easily deform (flow) under shear stress} and this property allows for the substance to adapt to containers or to move around other obstacles. An example is the air of the atmosphere, the liquid water in the sea, and even solids, like large accumulations of fine sand which subjected to vibrations can behave similar to liquid water.
Fluid mechanics is the science that studies everything related to the movement of fluids, like the forces involved or the properties of the fluid in its motion.

When a fluid moves around a solid (water around a rock), or the solid moves immersed in a fluid  (car moving through the air of the atmosphere) the fluid adapts to that relative motion by changing its properties, such as the velocity of its particles.
Where the fluid meets the solid, its particles have the same velocity as the solid because the forces in the interface fluid-solid (friction and pressure forces) are sufficient to deform the fluid.
The viscosity is the property of the fluid that opposes the deformation. Inside the fluid, the viscosity is seen as a force that reduces the velocity of the deformation or the velocity of the fluid particles.
In the example of the water in a river moving around a rock, far from the rock, the velocity of the water is that of the river, however the water in contact to the rock is not moving according to what was discussed earlier. The viscosity is responsible for adapting the velocity of the fluid from the velocity of the river to the velocity of the rock.
The change of velocity is actually done over a very small distance, and this region where the viscosity forces affect the fluid velocity is called boundary layer. 
The idea of the boundary layer was introduced by Ludwig Prandtl in (\citeyear{Prandtl04}) and it is the main topic of this thesis.

The boundary layers studied here are those pertaining to the momentum of the flow. The interest is in the effects of adverse pressure gradients in fully-developed turbulent boundary layers. To avoid the influence of other effects we consider the simplest scenario.
The fluid is isotropic, incompressible and Newtonian, therefore the temperature does not change the density and the viscosity of the fluid.

The simplest solid boundary is a flat plate without any heat or mass exchange (no suction or blowing). The adverse pressure gradient is then imposed as an outer boundary condition.
Comparisons can then be done with curved-boundary solids such as NACA wing profiles, where the pressure gradients are caused by the curvature of the profile. 
In these more realistic configurations we consider various mechanisms to control the flow, such as blowing or suction.

For these simple configurations we use a Cartesian system of coordinates where the space is denoted with $\pmb{x}=x_i=(x,y,z)$, which correspond to the vectorial, dyadic and coordinate notations that will be used in this work.
The boundary layers studied here are statistically two-dimensional (2D), $y$ is the wall-normal direction, $z$ is the homogeneous direction and $x$ is chosen as the streamwise direction. The time is denoted with $t$.
Although turbulence is a three-dimensional (3D) phenomenon, the flow under study is statistically 2D, since the $z$ direction is homogeneous.